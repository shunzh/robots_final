\documentclass[10pt]{article}
\usepackage{latexsym}
\usepackage{qtree}
\usepackage{algpseudocode}
\usepackage{natbib}
\usepackage{graphicx}
\usepackage{subfigure}
\title{Action Selection in ASAMI}
\author{Shun Zhang}
\date{}

\begin{document}
\maketitle

%\begin{abstract}
%\end{abstract}

\sloppy
\section{Literature Review}

In \cite{CSJ06}, action model and sensor model are both autonomously
learned by the robot in a bootstrapping way. Actions are selected
randomly in the training. However, this can be biased according to the
current belief.  In this sense, the agent should be able to determine
which action leads to most uncertain results and need more samples.
The agent doesn't know the correctness of the models. It only
knows the consistency of them. For example, states with larger
difference in action model and sensor model ($|W_a - W_s|$) should be
re-learned.  Unobserved states, even possibly predicted by the current
model, should also have higher priorities to be visited.

The first author of \cite{CSJ06}, Dan Stronger commented that "an
action could be causing problems is because the action model function
being fit, with the degrees of freedom that it has, just fits to a
function that's not especially accurate for that action".

This would a problem in degree selection in the polynomial regression.
Observed inconsistency can be caused by underfitting.  In this paper,
I'll give our action model proper degrees of freedom. So if the action
model is inconsistent, the reason should be either few data gathered
to make consistency happen, or the data gathered are noisy and not
making much sense.

Compared with this proposed method, some similar ideas are discussed
in the developmental robotics literature. Action selection using
Reinforcement Learning framework \cite{oudeyer2006discovering}
\cite{schmidhuber2006developmental}.


Two-dimensional ASAMI has been discussed in \cite{ICRA08-stronger}.

an instance-based action model is learned empirically by robots trying
actions in the environment \cite{LNAI2007-ahmadi}.

%=====================================================================
\bibliographystyle{abbrv}

\bibliography{report}

\end{document}
