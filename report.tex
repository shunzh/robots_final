\documentclass[10pt]{article}
\usepackage{latexsym}
\usepackage{qtree}
\usepackage{algpseudocode}
\usepackage{natbib}
\usepackage{graphicx}
\usepackage{subfigure}
\title{Action Selection in ASAMI}
\author{Shun Zhang}
\date{}

\begin{document}
\maketitle

%\begin{abstract}
%\end{abstract}

\sloppy
\section{Literature Review}

The action model and sensor model of a robot can be learned
simultaneously without external feedback \cite{CSJ06}.  In the
algorithm, called ASAMI, actions are selected randomly in the
training. However, this can be biased according to the current belief.
In this sense, the agent should be able to determine which action
would lead to most uncertain results and thus need more samples.  The
agent doesn't know the correctness of the models. It only knows the
consistency of them. For example, states with larger difference in
action model and sensor model ($|W_a - W_s|$) should be considered as
inconsistent.  Unobserved states can be also assumed as inconsistent.
In the perspective of model-adaptive agents \cite{maes1993modeling},
this tries to solve the problem of action selection.

The first author of \cite{CSJ06}, Dan Stronger, commented that ``one
possible reason the $W_a$ is wrong after a certain action is that the
action is just very noisy \ldots This is a good reason to gather more
data for that action \ldots  But another possible reason  an action
could be causing problems is because the action model function being
fit, with the degrees of freedom that it has, just fits to a function
that's not especially accurate for that action''. This would be a
problem in degree selection in the polynomial regression, which is
discussed in \cite{IJAIT08-stronger}. In this paper, I'll give our
action model proper degrees of freedom. So, if the action model is
inconsistent, the reason should be either few data gathered to make
consistency happen, or the data gathered are noisy and not making much
sense.

There is a two-dimensional version of ASAMI discussed in
\cite{ICRA08-stronger}.  As the action space is larger in higher
dimension, biasing actions instead of random selection on actions
becomes more essential. This can be proceeded if this proposal works
in one-dimensional environment.

Compared with this proposed method, some similar ideas are discussed
in the developmental robotics literature. They call the inconsistency
described above as \textit{error} \cite{oudeyer2006discovering},
\textit{surprise} \cite{ranasinghe2008surprise} or \textit{curiosity}
\cite{schmidhuber2006developmental}. This serves as a motivation to
change the model. A metric can be used to describe the global
inconsistency \cite{oudeyer2006discovering}. Then, a reinforcement
learning framework can be used to plan to get to the states with the
least global inconsistency. The authors in
\cite{ranasinghe2008surprise} used logic rules as the model, and let
the robot figure out why there could be a surprise. In this paper, I
assume that the surprise is caused by insufficient or noisy data, and
the knowledge is still the fitted functions. So the surprise, or
inconsistency, of an action, is naively caused by the data gathered at
that action.

There is an assumption in \cite{CSJ06} that there is a mapping from
action to the difference of state, i.e., $A \rightarrow \Delta S$.
This is true in robot motion. So it's possible to learn the difference
in the observation. This is exactly how the task is designed in
\cite{ICDL10-hester}. If this is not true, so that $S \times A
\rightarrow S$ is the best we can estimate, this would become more
challenging but still possibly to be dealt with. We might know that a
certain $(s, a)$ is inconsistent and we want to gather more data on
it. But our action model might not be well-learned. So the reward
should be a compromise between inconsistency of that $(s, a)$ and the
cost to reach it.

There are also other related works on learning the action model. But
they're using very different approach. They make use of the external
data and assume that the sensor model is well calibrated. The data are
represented as either statistics \cite{And_learningand} or instances
\cite{LNAI2007-ahmadi}.

%=====================================================================
\bibliographystyle{plain}

\bibliography{report}

\end{document}
